\documentclass{article}

%\usepackage[text]{these-pkg}
\usepackage{lipsum}
\usepackage{colortbl}

% \usepackage{amsfonts , amssymb , amsmath}
 \usepackage[francais,english]{babel}
% \usepackage[T1]{fontenc}
% \usepackage[utf8]{inputenc}
% \usepackage{lmodern}
% \usepackage{array,multirow,makecell,tabularx}
% \usepackage{tikz,graphicx}
% \usepackage[babel=true,kerning=true]{microtype}
% %\usetikzlibrary{shapes,snakes,decorations.text,calc}
% \usetikzlibrary{shapes,decorations.pathmorphing,decorations.text,calc,positioning}
% \usepackage{listings} % texte programmation 
% %\usepackage{cancel} % use of cancel command for feynman slash notation
% \usepackage{slashed} % better way to slash a character
% \usepackage[top=2cm, bottom=2cm, left=2cm, right=2cm]{geometry}
%  \usepackage{color}
 \usepackage{xcolor}
% \usepackage{colortbl}
% \usepackage[pdftex=true,colorlinks=true,linkcolor=blue,citecolor=red,
% filecolor=green,urlcolor=yellow]{hyperref}
% \usepackage{subfig}
% \usepackage{wrapfig}
% \usepackage{lineno}
% \usepackage{enumerate}
% \usepackage{appendix}
% \usepackage{perpage} %the perpage package
% \MakePerPage{footnote} % ==> permet de ré-initialiser le conteur des footnotes a chaque page

\begin{document}

\lipsum[1-10]

\begin{center}
  \begin{tabular}{l|l|l|}
    \cline{2-3}
    & \multicolumn{2}{c|}{Sensibilité} \\ \hline
    \multicolumn{1}{|c|} & \cellcolor[HTML]{F56B00}1.02 & 78 \\ \hline
  \end{tabular}
\end{center}

\lipsum[5-29]

\end{document}
