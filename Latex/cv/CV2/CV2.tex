%
\documentclass{cv}

\usepackage[francais]{babel} 
\usepackage[applemac]{inputenc}  %% les accents dans le fichier.tex
\usepackage[T1]{fontenc}       %% Pour la c�sure des mots accentu�s
\usepackage[paper=a4paper,textwidth=160mm,twosideshift=0pt]{geometry}

\newcommand{\lieu}[1]{{#1}\ }
\newcommand{\activite}[1]{\textbf{#1}\ }
\newcommand{\comment}[1]{\textsl{#1}\ }



\begin{document}

\begin{chapeau}
\begin{adresse}
	Isaac NEWTON\\%
	9.81, rue des Pommiers\\%
	Trinity College, Londres\\%
	\ligne\\%
	T�l. : 06 67 25 90 00\\%
	E-mail : \texttt{inewton@apple.com}
\end{adresse}
\begin{etatcivil}
	N� le 25/12/1642\\
	Nationalit� Anglaise
\end{etatcivil}
\end{chapeau}



	%%%%%%%%%%%%%%%%%%
	% Bloc rubriques %
	%%%%%%%%%%%%%%%%%%

\begin{rubriquetableau}[3.5cm]{Formation}

1665--1669 
	& \activite{Recherches � domicile}
	\comment{mention Tr�s Bien}
	\lieu{Lincolnshire}\\

1661--1665 
	& \activite{B.A. Degree, Math�matiques}
	\lieu{Universit� de Cambridge}\\

\end{rubriquetableau}

\begin{rubriquetableau}[3.5cm]{Activit�s Professionnelles}
1673--1683
        & \activite{Enseignement de l'Alg�bre}
        \lieu{(Trinity College, Cambridge)}\\
1665--1666
	& \activite{Observation de la chute des pommes}
	\lieu{Verger familial}\\


\end{rubriquetableau}

\begin{rubrique}{Langues} 
Anglais courant.
\end{rubrique}

\begin{rubrique}{Comp�tences en informatique}%
Syst�me : Linux

Langage : Pascal
\end{rubrique}


\end{document}
