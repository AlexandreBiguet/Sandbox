% This file is part of the "lettre" package.
% This package is distributed under the terms of the LaTeX Project 
% Public License (LPPL) described in the file lppl.txt.
%
% Denis M�gevand - Observatoire de Gen�ve.
%
% Ce fichier fait partie de la distribution du paquetage "lettre".
% Ce paquetage est distribu� sous les termes de la licence publique 
% du projet LaTeX (LPPL) d�crite dans le fichier lppl.txt.

\documentclass[12pt,origdate]{lettre}
\usepackage{epic,eepic}
\usepackage[francais]{babel}
\usepackage[OT1]{fontenc}
\usepackage{mltex}
\begin{document}
%
% Fichier de defaut de l'Observatoire
% ===================================
\institut{obs}
%
% Entete officielle, triple signature, format empty, date.
% ========================================================
% Lignes auxiliaires de reference et d'E-Mail. 
% ============================================
% Tests: quote, quotation, verbatim, minipage, 
% ============================================
%        picture, tabbing. 
% ========================
%
\begin{letter}{	Pr.~E.N.~Photon \\ 
                D\'epartement d'Astrotopographie \\ 
                Universit\'e de Saint Pozium \\
                3945, Quai du G\'eneral Gisant \\
                CH-6800 Motte-au-Rolla }

\pagestyle{empty}

\psobs
\name{Dr~D.P.~Dnavegem}
\location{Pr.~J.~Su\"{\i}jy-Rest\\Groupe des Forces Statiques}
\lieu{Sauverni}
\date{au 25 Joviet 2091}
\signature{Pr.~J.~Su\"{\i}jy-Rest\\ Doyen et\\ Chef de D\'epartement}
\secondsignature{Dr~D.P.~Dnavegem\\ Collaborateur Scientifique}
\thirdsignature{A.~Jout\'e\\ Assistant}

\vref{EP/mjs}
\nref{jsr002}
\username{jsr}
\bitnet{cgeuge54}
\decnet{chgate::20159}

\opening{Cher Professeur Photon,}

Nous vous remercions d'avoir donn\'e suite \`a notre requ\^ete, et vous
confirmons notre participation au symposium en tant que sp\'ecialistes
des affaires \'etranges. Veuillez trouver ici un r\'esum\'e de notre 
communication commune:

\begin{quote}\bf\sl
L'influence n\'efaste des extra-terrestres pendulaires et frontaliers
sur les communications t\'el\'evisuelles intercontinentales.
\end{quote}

\begin{quotation}
L'\'emergence de courants plasmuriques forts dans la r\'egion d'atterrissage 
des  v\'ehicules de liaison plan\'etaires (VLP) est \`a l'origine des champs 
gravito--organiques \`a bolomisations al\'eatoires connus depuis la fin du 
si\`ecle pass\'e. Ces modifications de l'\'equilibre physico-chimique de 
l'atmosph\`ere donnent lieu \`a toute une panoplie de ph\'enom\`enes plus 
ou moins inqui\'etants et spectaculaires, tels que les trous dans la couche 
d'ozone ou les aurores bor\'eales ou australes que l'on attribuait par le 
pass\'e \`a des regains d'activit\'e solaire.

On a d\'ecouvert r\'ecemment qu'aux heures de pointe, le flux des VLP, 
anciennement acronym\'es OVNIS,  pouvait provoquer des battements et des 
ph\'enom\`enes de r\'esonances dans certaines configurations de terrain, 
et sous certaines conditions, telles que les meilleurs blindages 
gravito--organiques ne pouvaient y \^etre totalement opaques. 

Les communications t\'el\'evisuelles intracontinentales, bas\'ees sur les 
technologies les plus r\'ecentes de fibres auditiques en 
Corduron$^{\mbox{\copyright}}$ de chez Dubond de Velours sont compl\'etement 
insensibles \`a de telles perturbations, contrairement aux anciennes lignes 
intercontinentales en cablage traditionnel (polygraphite impr\'egn\'e).  
\end{quotation}

Nous avons d\'evelopp\'e un logiciel d'analyse permettant de traiter 
l'information statistique fournies par les sondes FVLP, pour fournir 
\`a nos clients l'information sur les endroits les plus touch\'es du globe.
Veuillez trouver ci-apr\`es le pseudo-code du protocole de communication, 
une illustration des sondes se transmettant l'information de mani\`ere 
autonome, ainsi qu'une table des param\`etres de celles-ci.

\begin{verbatim}
BEGIN
   if(alive(S1) && alive(S3) && alive(S5)) then
      BEGIN
         contact{s1,S1};
         contact{S1,S3};
         contact{S3,S5};
         contact{S5,s3};
      END         
   endif
   if(alive(S2) && alive(S4) && alive(S6)) then
      BEGIN
         contact{s2,S2};
         contact{S2,S4};
         contact{S4,S6};
         contact{S6,s4};
      END         
   endif
END         
\end{verbatim}

\input{sondes.tex} % graphique en mode picture avec eepic

\begin{minipage}{7cm}
Les valeurs param\'etriques des satellites sont donn\'ees ci-contre, par ordre 
de date de lancement. Les unit\'es sont MKSA, dans la mesure du possible, 
l'excentricit\'e des orbites est donn\'e comme le rapport grand/petit axe, 
et le taux de transmission en TB/s.
\end{minipage}\hfill
\begin{minipage}{7cm}
\begin{tabbing}
n$^{\textrm o} $ \=masse \=g.a/p.a \=puissance \=t$_{\textrm tr}$\\
S1\>247\>1.16\>53.5\>1.3\\
S2\>211\>1.40\>49.3\>1.1\\
S3\>233\>1.27\>51.0\>1.2\\
S4\>199\>1.91\>48.8\>1.0\\
S5\>270\>1.33\>65.2\>1.5\\
S6\>270\>1.33\>65.2\>1.5\\
\end{tabbing}
\end{minipage}

\closing{Veuillez agr\'eer, Monsieur le professeur, l'expression 
         de nos condol\'eances distingu\'ees.} 

\end{letter}
%
\end{document}
