% This file is part of the "lettre" package.
% This package is distributed under the terms of the LaTeX Project 
% Public License (LPPL) described in the file lppl.txt.
%
% Denis M�gevand - Observatoire de Gen�ve.
%
% Ce fichier fait partie de la distribution du paquetage "lettre".
% Ce paquetage est distribu� sous les termes de la licence publique 
% du projet LaTeX (LPPL) d�crite dans le fichier lppl.txt.

\documentclass[12pt,origdate]{lettre}
\usepackage[francais]{babel}
\usepackage{epsf}
\usepackage[OT1]{fontenc}
\usepackage{mltex}
\begin{document}
%
% Fichier de defaut de l'Observatoire
% ===================================
\institut{obs}
%
% telefax, entete officielle
% ==========================
% champ Concerne, ligne d'E-Mail
% ==============================
%
\begin{telefax}{+41-22-320 29 27}{Universit\'e de Gen\`eve\\
                                Aux personnes concern\'ees}
\psobs
\location{Dr~D.~M\'egevand\\Ing\'enieur \TeX nicien}
\name{Denis M\'egevand}

\username{megevand}
\bitnet{cgeuge54}
\internetobs

\conc{Style T\'el\'efax pour \LaTeX}

\marge{5mm}
\opening{Chers coll\`egues}

J'ai modifi\'e le style \verb+LETTRE+ pour qu'il puisse cr\'eer une ent\^ete
plus adapt\'ee aux besoins du t\'el\'efax, et r\'epondre \`a vos attentes. 

L'ent\^ete est form\'ee ainsi que vous pouvez le voir ci-dessus d'un embl\`eme 
simplifi\'e de l'Observatoire, que vous pouvez bien sur modifier par la commande 
\verb+\address+. 

A sa droite se trouve le mot {\CMD T~\'E~L~\'E~F~A~X}, et au dessous une ligne 
toujours pr\'esente contenant par d\'efaut le num\'eros de t\'el\'ecommunications 
standard de l'Observatoire.
Vous pouvez modifier ces num\'eros par les commandes habituelles \verb+\telephone+, 
\verb+\fax+, \verb+\telex+. Vous noterez la disparition du num\'ero du telex de 
l'Observatoire, qui a \'et\'e retir\'e \`a la version 2.05.

Au dessous se trouve un champ en \'evidence contenant les informations propres
au t\'el\'efax: 
\begin{description}
\item[Destinataire]Ce champ provient de la commande \verb+\begin{telefax}+ qui
remplace le \verb+\begin+\-\verb+{letter}+, mais qui contient deux param\`etres 
au lieu  d'un seul. Le premier est le num\'ero de t\'el\'efax, et le second est 
le param\`etre classique de \verb+\begin{letter}+, contenant le nom et l'adresse.
Pour le t\'el\'efax que vous \^etes en train de lire, la commande est 
\verb+\begin{telefax}{+{\ttfamily+}\verb+41-22-755-3983}{Obser+\-%
\verb+vatoire de Gen\`eve\\+\-\verb+Aux personnes concern\'ees}+.
\item[Exp\'editeur]Ce champ provient de \verb+\location+, ou, si cette commande
n'existe pas, de \verb+\name+ qui est obligatoire.
\item[Nombre de pages]Ce param\`etre est calcul\'e automatiquement, mais oblige
l'utilisateur \`a \LaTeX\ er son document une deuxi\`eme fois pour r\'esoudre
la valeur correctement. Un message le signale \`a la fin de la compilation. La
commande \verb+\addpages{n}+ permet \`a \LaTeX de tenir compte de n pages 
suppl\'ementaires. {\bfseries Note}: \`A cause du calcul des pages, on ne peut pas
mettre plusieurs t\'el\'efax dans un m\^eme fichier.
\item[Remarque]Une ligne de remarque \`a l'intention du destinataire est 
plac\'ee en dessous, lui indiquant que faire en cas de mauvaise reception.
Cette ligne est en gros caract\`eres, et de ce fait comprise m\^eme si elle
est mal transmise.
\end{description}

En r\'esum\'e, les commandes utiles, sp\'ecifiques ou diff\'erement interpr\'et\'ees 
sont \verb+\begin+\-\verb+{telefax}+\-\verb+{num\'ero}+\-\verb+{nom\\+\-\verb+adresse}+, 
\verb+\end{telefax}+, \verb+\location+ et \verb+\addpages{n}+. Les autres commandes 
sont utilis\'ees comme pour une lettre.

De plus, la version 1.64 du 20.12.89 a apport\'e les quelques modifications suivantes:
\begin{description}
\item[Ent\^ete]On peut dessiner un embl\`eme officiel de l'Observatoire, avec 
l'\'ecusson de Gen\`eve comme ici,

\epsfbox[-55 0 55 55]{ecusson55.ps}

en utilisant la
commande \verb+\psobs+. Pour les t\'el\'efax, cet embl\`eme est \'egalement 
simplifi\'e. Cette commande n'est valable que si l'on travaille avec une
imprimante {\scshape PostScript}.
\item[e-mail]La ligne d'e-mail est s\'epar\'ee du corps de la lettre par un
trait horizontal.
\item[Marge]La marge peut \^etre modifi\'ee par la commande \verb+\marge{dimension}+, 
par d\'efaut on a \verb+\marge{15mm}+, ce qui la laisse comme avant.
\end{description}

J'aimerais conna\^{\i}tre vos r\'eactions, et avoir vos avis sur les traductions
des termes du t\'el\'efax. Cette version, num\'erot\'ee 1.70, est devenue version 
2.00 lorsque les traductions ont \'et\'e approuv\'ees, et le style stabilis\'e.

\closing{Meilleures salutations}

\end{telefax}
\vfill
\end{document}
