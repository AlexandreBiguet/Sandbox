\documentclass[10pt,xcolor=table]{beamer}

\usetheme{Boadilla}
\usecolortheme{default}

\setbeamerfont{title}{series=\bfseries,parent=structure}
\setbeamerfont{frametitle}{series=\bfseries,parent=structure}


\usepackage[francais]{babel}
\usepackage[T1]{fontenc}
\usepackage[utf8]{inputenc}
\usepackage{array,multirow,makecell,tabularx}
\graphicspath{{../images/}}
\usepackage{comment}
\usepackage{tikz}
\usepackage[beamer,customcolors]{hf-tikz}
\usepackage[babel=true,kerning=true]{microtype}

\usetikzlibrary{shapes,snakes,decorations.text}

\setcellgapes{1pt}
\makegapedcells
\newcolumntype{R}[1]{>{\raggedleft\arraybackslash }b{#1}}
\newcolumntype{L}[1]{>{\raggedright\arraybackslash }b{#1}}
\newcolumntype{C}[1]{>{\centering\arraybackslash }b{#1}}

\newcommand{\sensibilite}[1]
{
  \begin{minipage}{#1}
    \[
    \Sigma(X) =
    \frac{\sigma_X}{\bar{X}}\frac{1}{\sigma_{\mrm{rel}}^I}
    \left\{
      \begin{array}{@{}c@{}cc@{}}
        \Sigma & \le 1 & 
        \begin{array}{@{}l@{}}
          \textrm{Prédiction peu sensible}\\
          \textrm{à la valeur précise des} \\ 
          \textrm{inputs }\{m_\pi^*\,,f_\pi^*\,,c^*\}
        \end{array} \\ 
        && \\
        \Sigma & \gg 1 & 
        \begin{array}{@{}l@{}}
          \textrm{Prédiction très sensible}\\
          \textrm{à la valeur précise des }\\ 
          \textrm{inputs } \{m_\pi^*\,,f_\pi^*\,,c^*\}
        \end{array}\\ 
      \end{array}
    \right.
    \]
  \end{minipage}
}

\newcommand{\dispersion}[1]
{
  \begin{minipage}{#1}
    \[
    \sigma_{\mrm{rel}}^I = \frac{1}{3}\left[ 
      \frac{\sigma(m_\pi)}{\bar{m_\pi}} + 
      \frac{\sigma(f_\pi)}{\bar{f_\pi}} + 
      \frac{\sigma(c)}{\bar{c}} \right] \;
    \]
  \end{minipage}
}

\newcommand{\FTitle}{ \frametitle{Titre de la frame}}

\begin{document}

\begin{frame}
  \frametitle{Titre de la frame}
  \begin{tikzpicture}
    \begin{scope}[shift={(current page.south west)}]
      \draw[gray!50] (0,0) grid[step=2mm] (current page.north east);
      \draw[red!50] (0,0) grid[step=1cm] (current page.north east);
      \draw (0.2,1) node {1};
      \draw (0.2,2) node {2};
      \draw (0.2,3) node {3};
      \draw (0.2,4) node {4};
      \draw (0.2,5) node {5};
      \draw (0.2,6) node {6};
      \draw (0.2,7) node {7};
      \draw (0.2,8) node {8};
      \draw (0.2,9) node {9};
      \draw (1,0.5) node {1};
      \draw (2,0.5) node {2};
      \draw (3,0.5) node {3};
      \draw (4,0.5) node {4};
      \draw (5,0.5) node {5};
      \draw (6,0.5) node {6};
      \draw (7,0.5) node {7};
      \draw (8,0.5) node {8};
      \draw (9,0.5) node {9};
      \draw (10,0.5) node {10};
      \draw (11,0.5) node {11};
      \draw (12,0.5) node {12};
    \end{scope}
    \node[rectangle,very thick,anchor=south west,align=left,draw=red] 
  at (2,2) {Test positionnnement \\ d'un node dans une \\ frame} ;
\node[rectangle,very thick,anchor=north west,align=left,draw=red] 
  at (0,9) 
  {
    \begin{minipage}{7cm}
      \[
      \Sigma(X) =
    \frac{\sigma_X}{\bar{X}}\frac{1}{\sigma_{\textrm{rel}}^I}
    \left\{
      \begin{array}{@{}c@{}cc@{}}
        \Sigma & \le 1 & 
        \begin{array}{@{}l@{}}
          \textrm{Prédiction peu sensible}\\
          \textrm{à la valeur précise des} \\ 
          \textrm{inputs }\{m_\pi^*\,,f_\pi^*\,,c^*\}
        \end{array} \\ 
        && \\
        \Sigma & \gg 1 & 
        \begin{array}{@{}l@{}}
          \textrm{Prédiction très sensible}\\
          \textrm{à la valeur précise des }\\ 
          \textrm{inputs } \{m_\pi^*\,,f_\pi^*\,,c^*\}
        \end{array}\\ 
      \end{array}
    \right.
      \]
    \end{minipage}
  };
  \end{tikzpicture}
\end{frame}

\begin{frame}
  \begin{center}
    \begin{tikzpicture}
      \draw[help lines,xstep=.2,ystep=.2] (current page.south west) grid (current page.north east) ;
      %\foreach \x in {0,1,...,9} {\node [anchor=north] at (\x) {\x} ;}
       %\foreach \x in {0,1,2,3,4,5,6,7,8,9} { \node [anchor=north] at (\x/10,0) {0.\x}; }
       %\foreach \y in {0,1,2,3,4,5,6,7,8,9} { \node [anchor=east] at
        % (0,\y/10) {0.\y}; }
      %\foreach \x in {0,1,2,3,4,5,6,7,8,9} { \node [anchor=north] at (\x,0) {\x}; }
       %\foreach \y in {0,1,2,3,4,5,6,7,8,9} { \node [anchor=east] at (0,\y) {\y}; }
    \end{tikzpicture}
  \end{center}
\end{frame}


\end{document}

